%\chapter{Przykłady elementów pracy dyplomowej}
%
%\section{Liczba}
%
%Pakiet \texttt{siunitx} zadba o to, by liczba została poprawnie sformatowana: \\
%\begin{center}
%	\num{1234567890.0987654321}
%\end{center}
%
%
%\section{Rysunek}
%
%Pakiet \texttt{subcaption} pozwala na umieszczanie w podpisie rysunku odnośników do ,,podilustracji'': \\
%
%\begin{figure}[h]
%	\centering
%	\begin{subfigure}{0.35\textwidth}
%		\centering
%		\framebox[2.0\width]{A}
%		\subcaption{\label{subfigure_a}}
%	\end{subfigure}
%	\begin{subfigure}{0.35\textwidth}
%		\centering
%		\framebox[2.0\width]{B}
%		\subcaption{\label{subfigure_b}}
%	\end{subfigure}
%	
%	\caption{\label{fig:subcaption_example}Przykład użycia \texttt{\textbackslash subcaption}: \protect\subref{subfigure_a} litera A, \protect\subref{subfigure_b} litera B.}
%\end{figure}
%
%\section{Tabela}
%
%Pakiet \texttt{threeparttable} umożliwia dodanie do tabeli adnotacji: \\
%
%\begin{table}[h]
%	\centering
%	
%	\begin{threeparttable}
%		\caption{Przykład tabeli}
%		\label{tab:table_example}
%		
%		\begin{tabularx}{0.6\textwidth}{C{1}}
%			\toprule
%			\thead{Nagłówek\tnote{a}} \\
%			\midrule
%			Tekst 1 \\
%			Tekst 2 \\
%			\bottomrule
%		\end{tabularx}
%		
%		\begin{tablenotes}
%			\footnotesize
%			\item[a] Jakiś komentarz\textellipsis
%		\end{tablenotes}
%		
%	\end{threeparttable}
%\end{table}
%
%\section{Wzory matematyczne}
%
%Czasem zachodzi potrzeba wytłumaczenia znaczenia symboli użytych w równaniu. Można to zrobić z użyciem zdefiniowanego na potrzeby niniejszej klasy środowiska \texttt{eqwhere}.
%
%\begin{equation}
%E = mc^2
%\end{equation}
%gdzie
%\begin{eqwhere}[2cm]
%	\item[$m$] masa
%	\item[$c$] prędkość światła w próżni
%\end{eqwhere}
%	url = {https://www.researchgate.net/publication/224080530_Machine-learning-based_mechanical_properties_prediction_in_foundry_production}
%Odległość półpauzy od lewego marginesu należy dobrać pod kątem najdłuższego symbolu (bądź listy symboli) poprzez odpowiednie ustawienie parametru tego środowiska (domyślnie: 2 cm).


\chapter{Przygotowanie danych}

No i tutaj raporcik, jak przygotowywałem dane. A tu cytat: \cite{Santos09}).

A tutaj cytacik wikipedii \cite{wiki:klas.stat}.

3: \cite{Olson85}.

4: \cite{ferrite.meter}.

5: \cite{Babu13}

6: \cite{Saluja15}

7: \cite{Vitek03.I}

8: \cite{Vitek03.II}

9: \cite{Vasudevan13}

10: \cite{Bhadeshia07}

11: \cite{Badmos13}

12: \cite{Nieves09}

13: \cite{Yang16}

14: \cite{Yoseba08}

15: \cite{Azimi18}

16: \cite{Pauly16}

17: \cite{Britz17}

18: \cite{Shrestha13}

19: \cite{Wang20}

20: \cite{Durmaz21}

21: \cite{Stoll21}

22: \cite{Mitchell97}

23: \cite{Koza96}

24: \cite{Sawka20}

25: \cite{Ng}

26: \cite{Shorten19}

27: \cite{Gandhi21}

28: \cite{norma}

29: \cite{Sarkar18}

30: \cite{Bickel06}

31: \cite{Chmiel20}

32: \cite{Sawka18}

33: \cite{wiki:latex}

34: \cite{Boser92}

35: \cite{Grus18}

36: \cite{wiki:dec.drzewo}

37: \cite{Quinlan96}

38: \cite{wiki:dec.tree}

39: \cite{Ho95}

40: \cite{aporras}

41: \cite{Jaskowiak70}

42: \cite{Piryonesi20}

43: \cite{wiki:knn}

44: \cite{wiki:logit}

45: \cite{wiki:bayes}

46: \cite{Kurama20}

47: \cite{ColumbiaLearn}

48: \cite{Hur02}

49: \cite{Ramchandani18}

50: \cite{Menard02}

51: \cite{Piryonesi19}

52: \cite{Li}

53: \cite{Tadeusiewicz15}

54: \cite{Simonyan15}





















